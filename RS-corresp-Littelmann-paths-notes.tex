\documentclass[reqno]{amsart}
\usepackage{mathrsfs}
\usepackage{graphicx}
\usepackage{amscd}
\usepackage{amsmath}
\usepackage{amsthm}
\usepackage{amsfonts}
\usepackage{amssymb}
\usepackage{amsxtra}
\usepackage{xspace}
\usepackage{array}
\usepackage{cite}
\usepackage{color}
\usepackage{xypic}

\theoremstyle{definition}
\newtheorem{fact}{Fact}
\newtheorem{theorem}{Theorem}[section]
\newtheorem{corollary}{Corollary}
\newtheorem{lemma}[theorem]{Lemma}
\newtheorem{proposition}[theorem]{Proposition}
\newtheorem{remark}[theorem]{Remark}
\newtheorem{definition}[theorem]{Definition}




\newcommand{\lieg}{\mathfrak{g}}
\newcommand{\lieh}{\mathfrak{h}}
\newcommand{\integers}{\mathbb{Z}}
\newcommand{\complex}{\mathbb{C}}


\newcommand{\be}{\begin{enumerate}}
\newcommand{\ee}{\end{enumerate}}
\newcommand{\beq}{\begin{equation}}
\newcommand{\eeq}{\end{equation}}


\newcommand{\dom}[1]{\overline{#1}}
\DeclareMathOperator{\SYT}{SYT}
\DeclareMathOperator{\SSYT}{SSYT}
\DeclareMathOperator{\ssyt}{ssyt}
\DeclareMathOperator{\Yam}{Yam}
\DeclareMathOperator{\type}{type}


\begin{document}
\title{The RS correspondence, Littelmann paths, Schur-Weyl duality, etc}
\maketitle
\section{}
\subsection{}
Let $V =\complex^n$, $G = GL(V)$, $\lieg = \mathfrak{gl}(V)$. Let $\lieh \subseteq \lieg$ be the set of diagonal matrices. Let $\epsilon_i \;\;(1 \leq i \leq n)$ denote the $i^{th}$ diagonal element functional on $\lieh$. The weight lattice $X$ is the $\integers$-span of the $\epsilon_i$ and the set of dominant integral weights
\[X^+ = \{\sum_{i=1}^n \lambda_i \epsilon_i: \lambda_1 \geq \lambda_2 \geq \cdots \geq \lambda_n \geq 0\}\]
This set can thus be identified with the set of all partitions.

The dominant chamber is the positive cone generated by $X^+$.

\subsection{}
Consider the Littelmann path model for $V$ comprising the set of straight line paths from $0$ to $\epsilon_i$ for each $i$. This is the model obtained by choosing the initial (dominant) path to be the straight line to $\epsilon_1$ and acting on this by lowering operators $f_i$ in every possible way. Let us call these paths the {\em unit paths}.

%(this model corresponds to taking all tableaux with shape = a single box)

\subsection{}
Now let us take the tensor algebra $TV = \oplus_{d \geq 0}V^{\otimes d}$. The path model $\Pi(TV)$ for $TV$ (obtained from the chosen model for $V$) is thus the set of all finite concatenations of unit paths. This set will be identified with the set $\mathcal{A}^*$ of words in the alphabet $\mathcal{A}=\{1, 2, \cdots, n\}$ as follows:
\[ \mathcal{A}^* \to \text{ Path model for } TV ; \;\;\;\;\; w \mapsto \pi_w\]
where, given $w = i_1i_2\cdots i_p \in \mathcal{A}^*$, $\pi_w$ is the path
\[0 \longrightarrow \epsilon_{i_p} \longrightarrow \epsilon_{i_p} + \epsilon_{i_{p-1}} \longrightarrow \cdots \longrightarrow (\epsilon_{i_p} + \epsilon_{i_{p-1}} + \cdots + \epsilon_{i_1})\]
In other words, read the word $w$ in reverse, interpret each letter $i$ as the corresponding unit path, and concatenate.
By the {\em type} of $w$, we will mean the sum $\epsilon_{i_p} + \epsilon_{i_{p-1}} + \cdots + \epsilon_{i_1}$; this is simply the endpoint of $\pi_w$.

\subsection{Dominant paths, Yamanouchi words, and standard Young tableaux} A dominant path is one which stays entirely within the dominant chamber. Clearly, a path in $\Pi(TV)$ is dominant iff all its turning points are dominant weights. In other words, $\pi_w$ is a dominant path iff $\epsilon_{i_p} + \epsilon_{i_{p-1}} + \cdots + \epsilon_{i_j} \in X^+$ for each $j$; in other words iff $w$ is a {\em Yamanouchi} (or reverse-ballot) word.

For a given dominant weight $\lambda \in X^+$, the following sets have equal cardinality:
\be
\item The set of dominant paths in $\Pi(TV)$ with endpoint $\lambda$.
\item The set of Yamanouchi words of {\em type} $\lambda$.
\item The set $\SYT(\lambda)$ of standard Young tableaux of shape $\lambda$. 
\ee

To see this observe that if we identify $X^+$ with Young's lattice (the set of all partitions), then the sequence of turning points of the
dominant path $\pi_w$ is a path in Young's lattice from the empty partition $\phi$ to $\lambda$. This latter set is of course in bijection with $\SYT(\lambda)$.


\subsection{Yamanouchi words of a given type are Knuth equivalent}
\begin{fact}
  The set $\Yam$ of all Yamanouchi words is invariant under Knuth moves.
\end{fact}
This is a simply a matter of checking all 4 moves (K1, K2 and their inverses).
A point to note here is that the set of all ballot words is not Knuth invariant (?); so Yamanouchi-ness is more natural from this point of view.

So, $\Yam$ is a union of plactic equivalence classes. 
Now, by the Schensted insertion procedure, each plactic class contains a unique (reading word of a) tableau. 
\begin{fact}
If $w \in \Yam$ is the reading word of a tableau $\mathcal{T}$, then $\mathcal{T}$ is the {\em superstandard tableau} of type $= \type(w)$.
\end{fact}
The superstandard tableau has all 1's in the first row, all 2's in the second row, etc. This fact is straight-forward.

\begin{corollary}
  The set of Yamanouchi words of a given type forms a single plactic equivalence class.
\end{corollary}


\subsection{Plactic equivalence of paths}
Define two paths in $\Pi(TV)$ to be plactically equivalent if their corresponding words are plactically equivalent. Let's denote plactic equivalence by $\pi \sim \pi^\prime$.
Putting the previous subsections together, we have:
\begin{fact}\label{fact:dom-path-equiv}
  Two {\em dominant paths} $\pi_1$ and $\pi_2$ are plactically equivalent iff their endpoints are the same.
\end{fact}

Now Littelmann's crystal isomorphism theorem states that the corresponding path models are crystal isomorphic:
\begin{theorem}\label{thm:litt-crys-iso}
  Let $\pi_1$ and $\pi_2$ be dominant paths in $\Pi(TV)$ with the same endpoint $\lambda \in X^+$. Let $\Pi(\pi_j)$ be the path model generated by $\pi_j$, i.e, the set of paths obtained from $\pi_j$ by repeated application of the $f_{\bullet}$'s. Then, there is a (unique) bijection between $\Pi(\pi_1) \cup \{0\}$ and $\Pi(\pi_2) \cup \{0\}$ with the following properties:
\be
\item $\pi_1 \mapsto \pi_2$, $0 \mapsto 0$, and
\item If $\pi \mapsto \pi^\prime$, then $e_i \pi \mapsto e_i \pi^\prime$ and $f_i \pi \mapsto f_i \pi^\prime$, $\forall i, \; \pi \in \Pi(\pi_1), \pi^\prime \in \Pi(\pi_2)$.
\ee
\end{theorem}


\subsection{Crystal operators and plactic equivalence}
The following statement seems to follow from the Knuth relations, but the details have to be nailed down correctly (this must be there in some reference or the other, maybe Littelmann's Plactic paper ?)

\begin{theorem}\label{thm:crys-plac}
If $\pi \sim \pi^\prime$, then $e_i \pi \sim e_i \pi^\prime$ (in particular, if one of them is zero, then so is the other). Similarly, $f_i \pi \sim f_i \pi^\prime$. 
\end{theorem}

For a path $\pi$, define its {\em dominantization} $\overline{\pi}$ to be the unique {\em dominant path} obtained by acting repeated $e_\bullet$'s on $\pi$.
Fact \ref{fact:dom-path-equiv} and Theorems \ref{thm:litt-crys-iso}, \ref{thm:crys-plac} enable a characterization of plactic equivalence.

\begin{corollary}
Let $\pi, \pi^\prime$ be paths in $\Pi(TV)$ and let $\pi_1, \pi_2$ denote their {\em dominantizations}. Then $\pi \sim \pi^\prime$ iff the endpoints of $\pi_1$ and $\pi_2$ are equal and $\pi \mapsto \pi^\prime$ under Littelmann's crystal isomorphism map $\Pi(\pi_1) \to \Pi(\pi_2)$.
\end{corollary}

Thus, plactic equivalence simply means that $\pi^\prime$ is the shadow of $\pi$ in an isomorphic path model.


\subsection{The SSYT path model for $V(\lambda)$}
The SSYT path model corresponds to a ``canonical'' choice of initial dominant path with endpoint $\lambda \in X^+$; this initial path is taken to be the {\em superstandard path}, i.e, $\pi_{w^\circ}$ where $w^\circ$ is the reading word of the superstandard tableau of shape $\lambda$. Let $\SSYT(\lambda)$ denote the set of all paths in this path model. We know: \[\SSYT(\lambda) = \{\pi_\sigma: \sigma \text{ is the reading word of a  semistandard tableau of shape } \lambda\} \]
%We also set: \[\SSYT = \bigsqcup_{\lambda \in X^+} \SSYT(\lambda) \]

By the crystal isomorphism theorem, to each path $\pi \in \Pi(TV)$ there corresponds a unique path in the SSYT path model $V(\dom{\pi}(1))$. We denote this path $\ssyt(\pi)$.
\begin{fact}\label{fact:rs-corresp-paths}
The map: 
\begin{align}
  \Pi(TV) &\to \bigsqcup_{\lambda \in X^+} \SSYT(\lambda) \times \{\text{Dominant paths with endpoint } \lambda \}\\
  \pi &\mapsto (\ssyt(\pi), \dom{\pi})
\end{align}
is exactly the Robinson-Schensted correspondence, once we identify $\Pi(TV)$ with $\mathcal{A}^*$ and the set of dominant paths with paths in Young's lattice (which in turn are nothing but standard Young tableaux).
\end{fact}
We need to work out the precise details of the proof. But what's immediately obvious is that the map above is a bijection; this is because a path $\pi \in \Pi(TV)$ is uniquely determined by $\dom{\pi}$ and the sequence of lowering operators $f_\bullet$ such that $f_\bullet(\dom{\pi})=\pi$. The above map encodes the $f_\bullet$ via $\ssyt(\pi)$, since \[\ssyt(\pi) = f_\bullet(\text{superstandard path with endpoint } =\dom{\pi}(1))\]
(the image of $\pi$ under the crystal isomorphism $\Pi(\dom{\pi}) \to \SSYT(\dom{\pi}(1))$).

\subsection{Schur-Weyl duality and the decomposition of $V^{\otimes d}$}
Since $\Pi(TV)$ is a path model for $TV$, Littelmann's decomposition theorem states that:
\[ TV = \bigoplus_{\substack{\pi \in \Pi(TV)\\\text{dominant path }}} V(\pi(1)) \]
In fact paths with $d$-segments (i.e, $\pi_w$ where $w$ is a word of length $d$ in $\mathcal{A}^*$) form a path model for $V^{\otimes d}$; so the above equality holds degree by degree.

Recalling from the preceding sections that dominant paths in $\Pi(TV)$ can be identified with paths in Young's lattice and thereby with Standard Young tableaux, one obtains that the multiplicity of $V(\lambda)$ in the above decomposition is precisely $\#\SYT(\lambda)$. This is of course what Schur-Weyl duality gives us.








\end{document}
